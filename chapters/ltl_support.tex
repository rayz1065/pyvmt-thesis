\chapter{\ltl{} Support}
\label{ch:ltl-support}
\ltl{} properties aren't always supported directly by model checkers, since usually just safety and liveness properties are supported, which are also included within the core language of \vmtlib{}.
The importance of these two kinds of properties cannot be overstated, in fact their wide adoption is justified by the fact that they're sufficient even to verify more complex properties like those that can be written in \LTL{} as long as they're encoded properly.

Over the years multiple algorithms were developed to tackle this problem, two of which were chosen to be implemented as part of the thesis work.
These functionalities aim to assist the end user, offering a layer of abstraction that simplifies the model checking workflow, but they also serve as proof of concept to display the capabilities of this library and how it may be used to create reusable tools and implement complex algorithms in \python{} that don't depend on any underlying solver thanks to \pysmt{}'s formula manipulation tools.

\section{\ltltosmv{} Encoding}
One of the classic encoding procedures for \ltl{} was proposed in 1994 in a paper titled "Another Look at \ltl{} Model Checking" \cite{DBLP:conf/cav/ClarkeGH94}.
This procedure involves creating a tableau associated with the negation of the \ltl{} formula, finding a set of elementary subformulas and for each of them a condition that defines whether that subformula is satisfied (characteristic function, here referred to as sat value).
By composing the model with the tableau it's possible to analyze how the state of the \ltl{} specification changes as steps are performed on the original model.

Elementary subformulae are defined for the \ltl{} operators \fX{} and \fU{}, therefore the first step for encoding is normalizing the negated formula through the \texttt{LtlRewriter} walker which performs the following substitutions:
\begin{itemize}
    \item \fR{} $\longrightarrow{}$ $\lnot(\lnot \phi \U \lnot \psi)$
    \item \fF{} $\longrightarrow{}$ $\top \U \phi$
    \item \fG{} $\longrightarrow{}$ $\lnot (\F \lnot \phi)$
\end{itemize}

A specialized formula walker, \texttt{LtlEncodingWalker}, is then employed to compute all of the elementary subformulae and the associated sat values:
\begin{itemize}
    \item For \fX{} a fresh symbol \elx{} is created, with $\mathrm{sat}(\fX) = \elx{}$
    \item For \fU{} a fresh symbol \elu{} is created, with $\mathrm{sat}(\fU) = sat(\psi) \lor (\mathrm{sat}(\phi) \land \elu)$
\end{itemize}

To constrain the evolution of the new variables, a transition constraint is added for each of them:

\[ \elx{} \leftrightarrow (\mathrm{sat}(\phi))' \]

To ensure that the properties are eventually fulfilled, liveness properties are added to the model.
For every elementary subformula in the shape $\X (\fU)$, a property $\lnot \mathrm{sat}(\fU) \lor \mathrm{sat}(\psi)$ is added.
The properties are then combined into a single liveness property to ensure compatibility with the model checkers, which is negated to return to our original \ltl{} specification.

% Include sample implementation?

\section{\ltl{} Circuit Encoding}

A different approach for encoding was presented in "A circuit Approach to \ltl{} Model Checking" \cite{DBLP:conf/fmcad/ClaessenES13}.
This encoding works a little differently since it doesn't generate the tableau corresponding to the negated \ltl{} formula, it instead creates circuit-like monitors for each of the subformulae in the negated \ltl{} formula capable of verifying whether a counterexample to the specification exists.

For example, the specification $\X a \land (b \lor \F \lnot c) \U d$ is negated into $\X \lnot a \lor (\lnot b \land \G c) \R \lnot d$.
The formula is split into its subformulae, each of which is denoted by a variable creating an equivalent formula:
\begin{align}
           &     \mathit{z_0} \\
    \land\ & \G (\mathit{z_0} \rightarrow G c) \\
    \land\ & \G (\mathit{z_1} \rightarrow \lnot b \land \mathit{z_0}) \\
    \land\ & \G (\mathit{z_2} \rightarrow \mathit{z_1} \R \lnot d) \\
    \land\ & \G (\mathit{z_3} \rightarrow \X \lnot a) \\
    \land\ & \G (\mathit{z_4} \rightarrow \mathit{z_3} \lor \mathit{z_2})
\end{align}

For the process to work, the negated formula has to be yet again normalized, this time in \nnf{} or \NNF{}. \pysmt{} already has a formula walker to perform this rewriting but it doesn't support the custom \ltl{} operators, therefore an extension was required to add the following rewritings recursively:
\begin{itemize}
    \item $\lnot \fX$ $\longrightarrow$ $\X \lnot \phi$
    \item $\lnot \fG$ $\longrightarrow$ $\F \lnot \phi$
    \item $\lnot \fF$ $\longrightarrow$ $\G \lnot \phi$
    \item $\lnot (\fU)$ $\longrightarrow$ $\lnot \phi \R \lnot \psi$
    \item $\lnot (\fR)$ $\longrightarrow$ $\lnot \phi \U \lnot \psi$
\end{itemize}

A specialized formula walker called \texttt{LtlCircuitEncodingWalker} is used to find all of the elementary subformulae.
For every \ltl{} operator, as well as the \texttt{And} and \texttt{Or} operators, a \texttt{FreshSymbol} is created: this is the activator for the monitor, once it's verified the monitor will start monitoring the formula and outputting its state.
The output for this walker is the list of activators and the formula that the monitor observes.

Monitors have 3 outputs: \texttt{accept}, \texttt{failed}, and \texttt{pending}.
\begin{itemize}
    \item \texttt{pending} is used by the monitor to specify that it's waiting for a signal from the inputs;
    \item \texttt{failed} indicates that a violation of the specification has been detected, meaning that the trace is no longer a valid counterexample;
    \item If \texttt{accept} goes false forever it's an indication that the trace is not a valid counterexample.
\end{itemize}

The \texttt{LtlCircuitEncodingWalker} has a \texttt{make\_monitor} method that generates the outputs for the monitor, as well as the state variables and their constraints required by the monitor to function.
The exact values for the signals may be found in the mentioned paper, below the section `List of Monitors for \pltl'.
The initial activator $\mathit{z_0}$ is replaced by the variable $\mathit{is\_init}$ which is modeled to be true in the first cycle, thus the circuit starts observing the state of the specification right at the start of the model checking process.

After all of the monitors have been created there are just two more things to compute, the \texttt{has\_failed} condition which is set to true indefinitely when any of the monitors fails, and the \texttt{accept} condition which specifies that every monitor must output \texttt{accept} infinitely often in order for the trace to be a valid counterexample.
Finally, a single liveness property is generated, $\lnot \mathit{accept}$.
If this property is demonstrated to be true it means that a counterexample doesn't exist, which proves our original specification.

% Include sample implementation?

\section{Experimental Evaluation}

To validate the efficacy of this approach, an experimental evaluation was conducted on a set of benchmarks.
The time required by the encoding procedure is negligible compared to the time required by a solver to verify the validity of the generated property, what this evaluation focused on is therefore the quality of the generated encoding in terms of number of extra state variables required, and the time for solving.

\begin{figure}
    \includegraphics{ltlenc_time_comparison.png}
    \caption{Comparison between LTL2SMV and Circuit encoding in terms of time required for model checking.}
    \label{fig:ltlenc_time_comparison}
\end{figure}

The encoders were run on a set of almost 600 benchmarks, after the conversion \iceia{} was run to evaluate the efficacy of the encoding in terms of aiding the model checking process.
From the output of the solver, the \texttt{total\_time} was collected and used as a metric for comparison.
On the set of benchmarks that was analyzed, the \ltltosmv{} encoding generally performed better than the circuit encoding version, this can be seen in Figure \ref{fig:ltlenc_time_comparison}.

\begin{figure}
    \includegraphics{ltlenc_stvar_comparison.png}
    \caption{Comparison between LTL2SMV and Circuit encoding in terms of number of supplementary variables required to encode the property.}
    \label{fig:ltlenc_stvar_comparison}
\end{figure}

A possible explanation as to why the circuit encoding is slower can be found in Figure \ref{fig:ltlenc_stvar_comparison} which shows that this kind of encoding with the current implementation requires a much larger number of state variables compared to the \ltltosmv{} version.
It's possible to improve the efficiency of this algorithm by drastically reducing the number of auxiliary state variables required as suggested in its paper, by not creating a new state variable for each of the \texttt{And} and \texttt{Or} logic operators, instead using a single activator for formulae containing multiple of them.
\FloatBarrier{}

