\chapter{\vmtlib{} Support}
\label{ch:vmtlib}
\pyvmt{} allows for conversion to and from a standard specification format called \vmtlib{}, allowing for higher interoperability of the library with external systems.
This chapter will focus on the definition of the format and the possibilities unlocked by such a conversion.

\section{The \vmtlib{} Format}
\vmtlib{} is an extension of the \smtlib{} format and provides a standard interface for defining transition systems by exploiting the possibility to attach annotations to terms and formulas, which can be used to specify the model behaviors \cite{DBLP:journals/corr/abs-2109-12821}.

It is designed with a limited amount of language constructs, aiming for simplicity for anyone interested in implementing the language within their \vmt{} solver:
\begin{itemize}
    \item \textbf{:next name} is used to specify state variables. To define a state variable, two variables are created, one representing the current state and one representing the next state. The next annotation is used to link the next state variable to the current state variable.
    \item \textbf{:init} is used to specify the formula used to constrain the initial states for the system.
    \item \textbf{:trans} is used to specify the formula used to constrain the transitions from each state to the next for the system.
    \item \textbf{:invar-property idx} is used to specify an invariant property to be verified, in \ltl{} it's in the form \fsafe{} and requires a formula to always hold. A non-negative integer specifies the index of the property.
    \item \textbf{:live-property idx} is used to specify a live property to be verified, in \ltl{} it's in the form \flive{} and requires a formula to hold eventually hold forever. A non-negative integer specifies the index of the property.
\end{itemize}

% TODO show an example
An in-depth explanation of the format, including a list of available tools may be found on \vmtlib{}'s website \cite{VMT-LIB}.

\pyvmt{} modelling system is similar to that of \vmtlib{} making converting to and from the format simpler.
Despite that, it's important to note that this library doesn't just work as a wrapper for the format, since it also offers abstractions like the \texttt{Next} operator which are not supported by \vmtlib{} but they make working with \vmt{} problems simpler.
The conversion allows for higher interoperability since a model created in \pyvmt{} may be serialized to \vmtlib{} and used in a model checker which isn't supported directly by the library.

\section{Reading}
\pysmt{} already supports parsing \smtlib{} through the \texttt{SmtLibParser} class.
\pyvmt{} defines the \texttt{read} function which reads a script from an input stream, parses it, and runs through the all the annotated formulae to extract the model.
For every \texttt{:next} annotation a new state variable is added. Since \pyvmt{} only keeps track of curr state variables, the next state variables are ignored.
After that, any declaration which was not a curr or next state variable is considered as an input.

To make next state variables work properly in trans constraints and properties, a substituter is employed to replace every instance of a next state variable with the corresponding representation using the \texttt{Next} operator.

Reading an existing \vmtlib{} model can have several use-cases:
\begin{enumerate}
    \item Modifying a model by adding state variables or constraints in a programmatic way, or combining it with other models
    \item Analyzing an existing model (e.g. by creating a custom formula walker that reads the constraints, by printing some statistics,\dots)
    \item Running the same \vmtlib{} script against multiple solvers, with different combination of options
    \item \pysmt{}'s \texttt{SmtLibParser} shows exactly where parsing errors occur, helping debug broken \vmtlib{} files. This can be demoed by running \mintinline{bash}{python3 -m pyvmt < some_script.vmt}
\end{enumerate}

\begin{listing}
    \label{alg:reading-vmtlib}
    \inputmintedpy{py/vmtlib_support.py}{11}{38}
    \caption{A simple script which reads a \vmtlib{} model from the standard input, prints some statistics, modifies the model by adding a counter, and runs multiple model checkers over the modified model. The new model is then serialized to the standard output.}
\end{listing}

\section{Writing}
\label{sec:vmtlib-serialization}
\pysmt{} already provides printers for \smtlib{} supporting two formats: normal and daggified, the latter providing a version that avoids repeating common formulae.

The \texttt{serialize} method of the \texttt{Model} produces a \texttt{FreshSymbol} for each of the next state variables and uses them to replace all of the \texttt{Next} operators which are not supported in \vmtlib{}.
To make sure that the replacements work properly, a custom walker named \texttt{NextPusher} is used to push all of the \texttt{Next} operators down to the leaves, leaving them only on \texttt{Symbol}s.
Additionally, to replace the unsupported operators the \texttt{VmtLibSubstituter} is used. To be specific it performs a replacement of any next state variable with the corresponding \texttt{FreshSymbol}, leaving only supported operators in the final formula.

As previously mentioned, serializing a model to \vmtlib{} is useful whenever trying to use a model checker which is not directly supported by \pyvmt{}, but it can also be a step to use a model checker that doesn't support \vmtlib{} in the first place.
Once the model has been serialized, several tools are available to further convert the model to a different format, they may be found on the \vmtlib{} website under the section `Tool Support' \cite{VMT-LIB}, the currently supported formats for conversion are:
\begin{itemize}
    \item \btor{} \cite{DBLP:conf/cav/NiemetzPWB18} the format used by \boolector{}
    \item Constrained Horn Clauses
    \item the version of \smv{} used by \nuxmv{} \cite{DBLP:conf/cav/CavadaCDGMMMRT14}
    \item core \vmtlib{}, excluding language extensions
\end{itemize}

% \inputminted[firstline=37, lastline=38]{python3}{py/vmtlib_support.py}

\section{\ltl{}}
An extension of the \vmtlib{} format includes the possibility to specify \ltl{} properties through the \texttt{:ltl-property} annotation.

\pyvmt{} supports reading \ltl{} properties thanks to an extension of the \texttt{SmtLibParser} that adds the various operators as interpreted functions. Additionally, \ltl{} operators may be printed using the \texttt{VmtPrinter} and \texttt{VmtDagPrinter}, which respectively extend the \texttt{SmtPrinter} and \texttt{SmtDagPrinter} provided by \pysmt{}. The \fR{} operator is not natively supported by \vmtlib{}, properties that contain it must first be passed through the \texttt{VmtLibSubstituter} which replaces it with the equivalent $\lnot (\lnot \phi\ U \lnot \psi)$. As mentioned in the previous section this is done automatically while serializing a model.

Since \ltl{} properties are not part of the core language of \vmtlib{}, it's not expected that every model checker supports it natively.
To tackle this issue the property may be encoded into the model and replaced with a different kind of property, this will be the subject of Chapter \ref{ch:ltl-support}.

